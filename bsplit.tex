\documentclass[12pt]{article}
\author{Adam A. S. Green}
\title{Quantum Memory as a Beamsplitter}
\usepackage{amssymb, amsmath, hyperref}
\DeclareMathAlphabet{\mathscr}{OT1}{pzc}{m}{it}
\usepackage{epsfig, graphicx}
\usepackage{verbatim}
\begin{document}
\bibliographystyle{plain}

\maketitle
\section{Introduction}


\section{Demonstration}

Any quantum memory that has the `beamplitter' like hamiltonian will show this:
\begin{equation}
H_{\textrm{int}}= J^* a^\dagger b + J a b^\dagger
\end{equation}

where $a$ and $b$ obey bosonic communtation relations. On a higher level, this is a function of photon bunching, but there are several inherent limitations and interesting cases when a quantum memory is applied.

\section{CD Memory}
We can demonstrate photon bunching explicitly in the case of the CD memory\cite{arxiv}. This quantum memory is used because the author was familiar with the equations of motion.

We have previously derived the heisenburg equations of motion for the operators of interest, $E_{\textrm{out}}$ and $\sigma_{\textrm{z?}}$.

The heisenburg state we prepared for photon bunching is:
\begin{equation}
| \psi \rangle = \sigma_\textrm{z}(t=0)^\dagger \int d\omega \psi(w) E_0^\dagger(w) | 0 \rangle
\end{equation}

Where $E_0(\omega)$ is the creation operator of a photon of frequency $\omega$, $\psi(\omega)$ is the single-photon envelope in frequency space, and $\sigma_\textrm{z}(t=0)$ is the initial creation operator of a atomic excitation. In what follows, we will be assuming that a photon is already stored in the memory at $t=0$, and that another pulse is incident.

The three terms we are concered with are:
\begin{align}
\left| \langle \phi_{20}| \Psi \rangle\right|^2 &= \left|\langle 0 | \sigma_\textrm{z} \sigma_\textrm{z} | \Psi \rangle \right|^2\\
\left |\langle \phi_{11} | \Psi \rangle \right|^2 &= \int dt \left|\langle 0 | \sigma(t')  E_out(t) | \Psi \rangle \right|^2\\
\left |\langle \phi_{02} | \Psi \rangle\right|^2 &= \int dt \int dt'\left| \langle 0 |  E_\textrm{out}(t) E_\textrm{out}(t') | \Psi \rangle \right |^2
\end{align}

Where $| phi_{20}$ is a double excitation in the photon field, $| \phi{11} \rangle $ is photon and an atomic excitation, and $ | \phi_{02} \rangle $ is a double atomic excitation. 

\section{ $| \psi_{20} \rangle$}

\begin{equation}
\langle \phi_{20}| \Psi \rangle =\langle 0 | \sigma_\textrm{z} \sigma_\textrm{z} | \Psi \rangle
\langle \phi_{20}| \Psi \rangle =\langle 0 | \sigma_\textrm{z} \sigma_\textrm{z} 
\sigma_\textrm{z}(t=0)^\dagger \int d\omega \psi(w) E_0^\dagger(w) | 0 \rangle
\end{equation}

and the heisenburg equations for the operators $E_\textrm{out}$ and $\sigma_\textrm{z}$ are:
\begin{align}
E_\textrm{out}(t) &= E_\textrm{in}(t) + i \sqrt{\frac{2}{\kappa}} g(t) \sigma(t)\\
\sigma(t) &= \sigma(0) e^{-\tau} + i\sqrt{2} e^{-\tau} \int^\tau_w d t' e^\tau E_\textrm{in}(t) \frac{g(t)}{\sqrt{\kappa}}
\end{align}
Where $\tau = \int^t dt g^2(t)/\kappa$, $g(t)$ is a time-dependant coupling between the atomic ensemble and the cavity field, and $\kappa$ is the decay rate of the cavity. Inserting these equations we obtain:
\begin{multline}
\langle \phi_{20}| \Psi \rangle =\langle 0 | \left (  \sigma(0) e^{-\tau} + i\sqrt{2} e^{-\tau} \int^\tau_w d t' e^\tau E_\textrm{in}(t) \frac{g(t)}{\sqrt{\kappa}} \right )\\ \left ( \sigma(0) e^{-\tau} + i\sqrt{2} e^{-\tau} \int^\tau_w d t' e^\tau E_\textrm{in}(t) \frac{g(t)}{\sqrt{\kappa}} \right ) \sigma_\textrm{z}(t=0)^\dagger \int d\omega \psi(w) E_0^\dagger(w) | 0 \rangle
\end{multline}

We know that $E_\textrm{in}$ is the input-pulse, and can also be defined in frequency space as $E_\textrm{int} = \int dw E_0(\omega) $ and furthermore, $[E_0(\omega), E_0^\dagger(\omega') ] = \delta(\omega-\omega')$, thus we know that 
\begin{equation}
E_\textrm{in} \int dw' \phi(\omega') E_0^\dagger(\omega') = \mathscr{F}[\phi(\omega)]
\end{equation}

And we know that the only terms that will survive in $\langle \phi |$ must contain $E_in(t) sigma(t)$. This leaves the following expansion:
%\begin{align}
%\langle \phi_{20}| \Psi \rangle &= 2^{\frac{3}{2}} i e^{-2\tau} \sigma(0) 
%int^\tau' E_{in}(\tau') \frac{g(\tau')}{\sqrt{\kappa}} e^{w\tau'} \sigma^\dagger(0) \int dw' E_0(\omega') \psi(\omega')\\
%\end{align}
%&= 2^{\frac{3}{2} i e^{-2 \tau} \int d \tau' e^{\tau'} \chi(\tau')
%\end{align}
where $\chi(\tau') = \mathscr{F}(\psi(\omega)) \frac{\sqrt{\kappa}}{g(\tau')}$

Now, we can also make a simplifying assumption, by noting that the optimal read-in condition is met when $\mathscr{F}(\psi(\omega)) = \sqrt{\frac{2}{\kappa}} g(t) e^{\tau(t)}$ and under this condition is met, the above equation reduces to:
\begin{align}
\langle \phi_{20}| \Psi \rangle & = 2^{\frac{3}{2}} i e^{-2\tau} \int d \tau' e^{2 \tau} \\
& = 2^{\frac{3}{2}} i\left(1- exp{-2\tau}\right)
\end{align}

And to be in the 50-50 beamsplitter regime, we know that this has to be equatl to i$\frac{1}{\sqrt{2}}$, so we can solve for $\tau$ for this to be true. In fact, it can be seen that if $\tau = ln(\sqrt(2)$ then the above equation reduces to:
\begin{align}
\langle \phi_{20}| \Psi \rangle &= 2^{\frac{3}{2}} i \frac{1}{2} \\
&= \frac{1}{\sqrt{2}}
\end{align}
\end{document}

